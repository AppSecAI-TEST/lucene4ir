%!TEX root = lucene4IR2016workshop_report.tex

\subsection{Challenges and Opportunities}
During the workshop the challenges and opportunities between academia and industry were discussed, focusing on research, teaching and learning, and graduate attributes. Below is a summary of the main points stemming from the discussion.

The first point discussed were research opportunities and challenges that can arises between industry and academia in information retrieval, text mining and big data domain. One of the first problems discusses was regarding the access to data and research problems. First, not having access to data often precludes academic investigations,  but even when available, the data needs to be of high enough quality and representative of the research problems faced by the company, for useful solutions to be developed. It was noted that in industry they often hit upon really interesting problems, but often, do not have the time, to investigate - it was suggested that this is where academia could potentially help - big problems are hard to find - however, finding funding to help companies (especially small companies) work with academia was seen as difficult to acquire and very time consuming with high overheads - and so more efficient knowledge transfer mechanisms were needed (at funding body level). Another aspect regarding the data, is that often it difficult to disclosure data because Non-Disclosure Agreements (NDA) are in place with clients or the data is in-house. While one solution would be if the client would agree to its release (unlikely), or that a research project between the industrial partner, the client and the academic be formed (long lead time). Another alternative suggested would be to abstract the problem away from the data and client so that the problem can be exposed - without disclosing the dat or compromising the client. While this means that the data is still an issue - but at least then the problem can be further examined - and funding sought to solve it.



%More specifically to the field of search and big data, academic participants felt that there was a lack of good and big real-world test data for researchers. 
%They expected companies to share data with that quality that is representative enough of the current challenges companies facing. 
%In turn, academia can provide solutions which can be feed back to these companies as a potential solution that can be deployed by them in the future. 

%It was noted communicating current challenges or sharing data for companies might be difficult due to the Non-disclosure agreement (NDA) they have with their clients, at least this is the case for small open sourced companies. 
%The industry participants mentioned that a scenario in which this situation can be realised is that a client come forward with a funding for investing in solving a challenging problem that requires an academic to do so, but this does not happen that often. 

The next point discussed were the need for a common open source platform that can be used for teaching and learning information retrieval and big data. 
For example, both academic and industry participants agreed upon the need for a better and general documentation for Lucene which is not bound to any specific features that is version dependent. 
The industry participants felt that such teaching and learning materials should be developed by academics. 
It was also noted the possibility of a collaboration between academia and industry via a funding scheme such as the Knowledge Transfer Partnerships\footnote{\scriptsize{\url{http://www.esrc.ac.uk/funding/funding-opportunities/knowledge-transfer-partnerships/}}}. 
However, the industry participants mentioned that they expected academia take the lead on this since they have the required expertise to apply for grant applications. 

The final point discussed were the graduate attributes and the skills required by Computer Scientist and Software Engineering Graduates working in information retrieval, text mining and big data. 
The obvious and core skills required in terms of being able to : (i) develop high quality, robust code, (ii) understand and think about complex systems/problems, (iii) communicate, discuss and resolve issues and (iv) have a good understanding of software engineering principles and practices, were seen as mandatory. 
More specifically to the field of search and big data, industry participants felt that there was a lack of skill graduates that knew about search technologies, how to process large scale data sets, and how to work with big (text) data. 
They expected candidates to understand more about the processes and core concepts of search and big data - to be able to describe it at a conceptual level, at least - but with ideally some practical skills (i.e. Lucene, Solr, Spark, Pig, etc). 
It was noted that the learning curve can be very steep when picking such technologies - but such skills are seen to be increasingly valuable by employers. 
For example, Lucene is a very large and complex project, that has evolved over years, so it can feel very opaque and daunting to begin with, however, given that it one of the largest OS toolkits for information and data mining and retrieval it is a skill worth learning. 
It was felt that there was a strong need for developing more training and resources for beginners to learn how to use such toolkits - and this was seen to be an area where more investment from funding agencies and universities could be directed. 






%In addition, many open source search companies are small and therefore they find it hard to support apprenticeships, internships, etc. 

