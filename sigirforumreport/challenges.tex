%!TEX root = lucene4IR2016workshop_report.tex
\subsection{Challenges and Opportunities}


During the workshop the challenges and opportunities between academia and industry were discussed. 
Here we are providing a summary of the main points.

The first challenge is the lack of good real-world test data for researchers. This is something that academics expects companies to provide. 
Also, if the industry can communicate current challenges they had, academics can provide solutions which can be feed back to industry as a potential solution in the future. 
For example, industry needs internal indicators of utility within an organisation of search to measure loss or gain. 
Such metrics can be developed by academics and provided to industry. 
In particular when the search domain is getting further apart from traditional ``Site Search'' and goes towards ``Enterprise Search'' which the data set are not that massive, hence ``Google'' approaches might not be applicable. 
However, the only scenario in which this situation can be realised is that a client come forward with a funding for investing in solving a challenging problem that requires an academic to do so, but this does not happen that often. 
In other scenarios, from industry point of view, communicating their current challenges might be difficult due to the Non-disclosure agreement (NDA).

It was also mentioned that there is a need for a common open source platform that can be used for teaching and learning purpose. 
In order to use Lucene for this purpose, there is a need for a better and general documentation which is not bound to any specific features that is version dependent. 
But the question was ``whether documentation should come from academia to industry or vice versa?''
It seems like industry people don't have time to create the required documentation and they would expect academics to provide that. 

Therefore, there is a need to access funds to create teaching and resource material as well as contribute to the open source materials. 
This leads us to the next challenge which is ``how industry and academics can collaborate using some funding like knowledge transfer?''
It was pointed out that there is time pressure and lack of expertise to apply for grant applications in the industry and academics can provide the required expertise on this matter. 
Also there is a huge amount of paperwork involved in getting a grant which makes it unappealing for companies. 

The final point discussed was the characteristics of a good candidate to be employed in search-based companies.
A good candidate is expected to be able to develop code, have knowledge about search, have the ability to think in a search focused way and diagnose problems.
Candidates are expected to know the process and core concepts and be able to describe it at a conceptual level.  
However, it was mentioned that Lucene community can be hard to enter, the learning curve can be steep, since this is a very large and complex project. 
In addition, many open source search companies are small and therefore they find it hard to support apprenticeships, internships, etc. 

