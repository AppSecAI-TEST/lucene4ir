%!TEX root = Lucene4IR2016workshop_report.tex
\subsection*{Keynote Talk: Apache Lucene in Industry} 
{\bf Charlie Hull, Flax}: In his talk, Charlie first introduced Flax, and how it evolved over the years. Charlie explained that they have been building search applications using open search software since 2001. Their focus is on building, tuning and supporting fast, accurate and highly scalable search, analytics and Big Data applications. They are partners with Lucidworks, leading Lucene specialists and committers. When Lucene first came out clients were at reluctant to adopt open source, but nowadays it has been much more acceptable. Charlie notes that now you don't have to explain to clients what open source software is, and why it should be used. He described how Lucene-based search engines have risen in use - and that search and data analytics are available to those without six figure budgets. Charlie points out that Lucene is appealing because it is the most widely used open source search engine, which is hugely flexible, feature rich, scalable and performant. It is supported by a large and healthy community and backed by the Apache Software Foundation. Many of world's largest companies use Lucene including Sony, Siemens, Tesco, Cisco, Linkedin, Wikipedia, WordPress and Hortonworks. Charlie notes that they typically don't use Lucene directly, instead they use the search servers, built on top of Lucene, i.e. Apache Solr (which is mature, stable, and crucially highly scalable), or ElasticSearch (easy to get started with, great analytics, scalable). He contrasts these products with some of the existing toolkits in IR, and remarks on the latter, that ``no one in industry has ever heard of them!''. So even though they have the latest research encoded within them, it is not really viable for businesses to adopt them, especially as support for such toolkits is highly limited. He recommends that IR research needs to be within Lucene-based search services for it to be used and adopted. 

%Charlie then described a number of projects that they have been working on. (1) and (2)..

Based on Charlie's experience he provided us with a number of home truths:
\begin{itemize}
	\item Open source does not mean cheap 
	\item Most Search engines are the same (in terms of underlying features and capabilities)
	\item Complex features are seldom used - and often confusing
	\item Search testing is rarely comprehensive
	\item Good search developers are hard to find
\end{itemize}


Charlie reflected on these points considering how we can do better. First, learn what works in industry and how industry are using search - there are lots of research challenges which they rarely get to solve and address but solutions to such problems would have real practical value. Second, improve Lucene et al with ideas from academia - faster - for example, it took years before BM25 replaced TFIDF as the standard ranking algorithm, where as toolkits like Terrier already have infrastructure for Learning to Rank, while this is only just being developed in Lucene. Third, he pointed out that testing and evaluation of Lucene based search engines is very limited, and that thorough evaluations by search developers is poor. He argued that this could be greatly improved, if academics and researchers, contributed to the development of evaluation infrastructure, and transferred their knowledge to practitioners on how to evaluate. Lastly, he pointed that the lack of skilled and knowledgable search developers was problematic - having experience with Lucene, Solr and ElasticSearch are highly marketable skills, especially, when there is a growing need to process larger and larger volumes of data - big data requires data scientists! So there is the pressing need to create educational resources and training material for both students and developers. 
