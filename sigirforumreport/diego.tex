%!TEX root = lucene4IR2016workshop_report.tex
\subsection*{Learning to Rank with Solr} 
{\bf Diego Ceccarelli, Bloomberg}
On day two of the workshop, Diego started his talk by explaining that tuning Lucene/Solr et al. is often performed by ``experts'' who hand tune and craft the weightings used for the different retrieval features. However, this approach is manual, expensive to maintain, and based on intuitive, rather than data. His working goal behind this project was to automate the process. He described how this motivated the use of Learning To Rank, a technique that enables the automatic tuning of an information retrieval system by applying machine learning when estimating parameters. He points out that sophisticated models can make more nuanced ranking decisions than a traditional ranking function when tuned in such a manner. During his talk, Diego presented the key concepts of Learning to Rank, how to evaluate the quality of the search in a production service, and then how the Solr plugin works. At Bloomberg, they have integrated a learning to rank component directly into Solr (and released the code as Open Source), enabling others to easily build their own Learning To Rank systems and access the rich matching features readily available in Solr. 


