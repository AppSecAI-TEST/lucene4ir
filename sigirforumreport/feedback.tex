%!TEX root = lucene4IR2016workshop_report.tex
\subsection{Feedback}

At the end of the workshop we asked participants to answer the following question: "What would you suggest us to Stop/Start/Continue to do in our next workshop?". 
Here we present main points derived from this feedback. 
First of all, participants were pleased by the workshop, specially they enjoyed the hackathon part as well as the talks from the industry. 
They also encouraged us to organise a follow up workshop on this topic.   
Participants liked the idea of the workshop were it brought together academics and industry and encouraged us to continue inviting people from industry and in particular Lucene developers.
They also suggested to Invite more undergraduate students to the talks and hackathon.  

In addition, participants suggested that they would prefer to have a more well-defined structured hackathon, which means, setting more well-designed goals, having tools and data tested and ready, as well as making sure all participants use the same data. 
Some participants suggested defining the goals and objectives by asking participants before the hackathon. 
One proposed idea was to create a standardised mini-competition between workshop participants, e.g. providing a template code with a challenge to increase MAP with the expectation that each team formally presents its results.
Another proposed idea was to organise a mini-tutorial on Lucene features (related to the hackathon), or provide active demos of Lucene module implementation. 
A related proposed idea was to provide a basic quick-start workshop and documentation for people unfamiliar with Lucene. 
Participants also requested for shorter talks as well as recording and streaming them. 
They also requested to have slightly more technical talks focusing on Lucene architecture. 
Finally, some participants were requesting to include Solr into the program and promote the event in the Lucene and Solr Communities.
