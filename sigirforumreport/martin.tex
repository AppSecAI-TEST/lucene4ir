%!TEX root = lucene4IR2016workshop_report.tex
{\bf Martin Halvey, University of Strathclyde}: The Department of Computer and Information Science at the University of Strathclyde has two modules relevant to the discussion. The first module is Information Access \& Mining (ISA) is delivered to final year undergraduates and cover a range of techniques for extracting information from textual and non-textual resources, modelling the information content of resources, detecting patterns within information resources and making use of these patterns. The second is Information Retrieval and Access (IRA) which is delivered to Masters students. This module is a required module for students on Strathclyde's Information \& Library Studies and Information Management Masters Programmes, as well as being an optional module for other Masters students. To offer a contrast to other modules Martin described IRA in detail, as the cohort is different to others described. Typically, with some exceptions, the students do not have experience in programming or mathematics in their undergraduate degree. This presents a number of challenges when teaching some of the core concepts, where the syllabus includes:
information seeking and behaviour,
indexing,
term weighting,
retrieval models,
IR evaluation,
multimedia retrieval,
user interfaces and interaction,
Web retrieval. 

Martin explained that in laboratory and tutorial sessions that students were given problems to solve on paper. The intention is that students understand how different concepts, models, evaluation measures etc. work. For some problems students are provided with spreadsheets that automatically calculate some of the equations discussed in lectures so that students can see the relationship between different inputs and outputs. 

Martin outlined how developing some demonstrators using Lucene could replicate what he currently does with spreadsheets, with the benefit being that these demonstrators would be based on a real toolkit and also be more adaptable to use a wider range of retrieval models, evaluation measures etc. There is also the possibility in future years that students will be introduced to tools like Apache Lucene and ElasticSearch in a different module to IRA. Here, it was pointed out by Ian Ruthven, that often these students wont need modify such toolkits, but they will need to know how the systems work, how to configure them, and how to evaluate their configurations choices.\\

