%!TEX root = lucene4IR2016workshop_report.tex
\subsection*{Black Boxes are Harmful}

{\bf Sauparna Palchowdhury (National Institute of Standards and
  Technology, Gaithersburg, Maryland, USA)}: Having seen students and
practitioners in the IR community grapple with abstruse documentation
accompanying search systems and their use as a black box, Sauparna, in
his talk, argued why Lucene is a useful alternative and how and why we
must ensure it does not become another black box. In establishing his
views, he described the pitfalls in an IR experiment and the ways of
mitigation. The suggestions he put forth, as a set of best practices,
highlighted the importance of evaluation in IR to render an experiment
reproducible and repeatable and the need for a well-documented system
with correct implementations of search algorithms that are traceable
to a source in IR literature. In the absence of such constraints on
experimentation students are misled and learn little from the results
of their experiments and it becomes hard to reproduce the
experiments. As an example, the talk cited a wrong implementation of
the \emph{Okapi BM25} term-weighting equation in a popular research
retrieval system (Table \ref{tab:tfxidf}). Following this was a brief
how-to on implementing \emph{BM25} (or any TF$\times$IDF weighting
scheme) in Lucene (Table \ref{tab:lucene}). This also explained
Lucene's way of computing the similarity between two text documents
(usually referred to as \emph{Lucene's scoring
  formula}\footnote{\url{https://goo.gl/ZOMVYe}}).

Some of the points of failure mentioned in the talk were misplaced
test-collection pieces (document-query-qrel triplet), counterintuitive
configuration interfaces of systems, poor documentation that make
systems look enigmatic and lead to the creation of heuristics passed
around by word-of-mouth, naming confusion (a myriad of TF$\times$IDF
model names), blatant bugs and a obscure parser. As mitigation,
Sauparna listed some of the things he did as an experimenter. He wrote
a script (TRECBOX\footnote{\url{https://github.com/sauparna/TRECBOX}})
to abstract parts of the IR experiment pipeline and map them to
configuration end-points of the three systems; Terrier, Lucene and
Indri. This would enable documenting and sharing an experiment's
design in plain text files. He constructed a survey of term-weighting
equations titled \emph{TF$\times$IDF
  Repository}\footnote{\url{http://kak.tx0.org/IR/TFxIDF}} meant to be
a single point of reference to help disambiguate the variants in the
wild. All equations mentioned in this repository are traceable to a
source in IR literature. He also showed how to visually juxtapose
evaluation results obtained using a permutation of a set of systems,
retrieval models and test-collections on a chart that would act as a
sanity check for the system's integrity. As a part of these
investigations he modified Lucene for use with TREC collections (the
mod was named LTR\footnote{\url{https://github.com/sauparna/LTR}})
which is available for others to use. The ``mod'' is also accompanied
by notes to augment Lucene's documentation. The gamut of Sauparna's
work is collected on a website\footnote{\url{http://kak.tx0.org/IR}}.

Lucene's documentation does not use a well-defined notation to
represent its way of computing the similarity score between a query
$Q$ and document $D$. Equation \eqref{eqn:Lucene-scoring} denotes
Lucene's scoring formula as described in Lucene's documentation. In
the equation, $T$ denotes a term. The functions, in order from left to
right, on the right-hand-side of the equation is the
\emph{coordination factor}, \emph{query normalization factor},
\emph{term-frequency transformation}, \emph{document-frequency
  transformation}, \emph{query boost} and \emph{document-length
  normalization factor}. A well-defined, generalized, notation for
Lucene's scoring, in step with the definition from Lucene's
documentation, is Equation \eqref{eqn:Lucene-scoring-generalized}
(function names were shortened appropriately).

\begin{equation}
  score(Q,D) = coord(Q,D) \cdot qnorm(Q) \cdot \displaystyle\sum_{T \in Q} (tf(T \in D) \cdot idf(T)^2 \cdot boost(T) \cdot norm(T,D))
  \label{eqn:Lucene-scoring}
\end{equation}

\begin{equation}
  score(Q,D) = f_{c}(Q,D) \cdot f_{q}(Q) \cdot \displaystyle\sum_{T \in Q \cap D}(tf(T) \cdot df(T) \cdot f_{b}(T) \cdot f_{n}(T,D)))
  \label{eqn:Lucene-scoring-generalized}
\end{equation}

To explain Lucene's scoring, Sauparna picked two popular TF$\times$IDF
variants, broke them down into meaningful components (a term-frequency
transformation, a document-frequency transformation and a
length-normalization coefficient) and plugged these components into
Lucene's equation. The components in Lucene's equation that were left
unused were replaced by the integer $1$, meaning, the functions
returned $1$; which would have no effect on the chain of
multiplications. Table \ref{tab:tfxidf} lists the variants and
components and Table \ref{tab:lucene} shows where the components were
transplanted to.

\begin{table}
  \centering
  \small
  \begin{minipage}[t]{0.65\textwidth}
    
    \begin{tabular}{lcc}
      \multicolumn{3}{c}{TF$\times$IDF Variants: What's correct and what's not.}\\
      \\
      Name & $w_{ik}$ & $w_{jk}$\\
      \hline\hline
      \\
      \emph{BM25}(A)
      & $\frac{f_{ik}}{k_{1}((1-b)+b\frac{dl_{i}}{avdl})+f_{ik}} \times \log(\frac{N-n_{k}+0.5}{n_{k}+0.5})$
      & $\frac{(k_{3}+1)f_{jk}}{k_{3}+f_{jk}}$ \\
      \\
      \emph{BM25}(B)
      & $\frac{(k_{1}+1)f_{ik}}{k_{1}((1-b)+b\frac{dl_{i}}{avdl})+2f_{ik}} \times \log(\frac{N-n_{k}+0.5}{n_{k}+0.5})$
      & $\frac{(k_{3}+1)f_{jk}}{k_{3}+f_{jk}}$ \\
      \\\hline
      \\
      \emph{Okapi BM25}
      & $\frac{(k_{1}+1)f_{ik}}{k_{1}((1-b)+b\frac{dl_{i}}{avdl})+f_{ik}} \times \log(\frac{N-n_{k}+0.5}{n_{k}+0.5})$
      & $\frac{(k_{3}+1)f_{jk}}{k_{3}+f_{jk}}$ \\
      \\
      components & $TF \times DF$ & $QTF$ \\
      \\\hline
      \\
      \emph{SMART dtb.nnn}
      & $\frac{(1+\log(1+\log(f_{ik}))) \times \log(\frac{N+1}{n_{k}})}{1-s+s \cdot \frac{b_{i}}{avgb}}$
      & $f_{jk}$ \\
      \\
      components & $TF \times DF \div LN$ & $QTF$ \\
      \\\hline\hline

    \end{tabular}
    
    \caption{ \small The similarity score;
      $score(D_{i},D_{j})=\sum_{k=1}^{t}(w_{ik} \cdot w_{jk})$
      $\forall i \neq j$, combines the weight of a term $k$ over the
      $t$ terms which occur in document $D_{i}$ and $D_{j}$. Since a
      query can also be thought of as a document in the same vector
      space, the symbol $D_{j}$ denotes a query. \emph{BM25}(A) and
      \emph{BM25}(B) are the two incorrect implementations found in a
      popular retrieval system. Comparing them to \emph{Okapi BM25} on
      the third row shows that A has the $k_{1}+1$ factor missing in
      the numerator, and B uses twice the term-frequency, $2f_{ik}$,
      in the denominator. Neither can they be traced to any source in
      IR literature, nor does the system's documentation say anything
      about them. The \emph{Okapi BM25} and the \emph{SMART dtb.nnn}
      variants are known to be effective formulations developed by
      trial and error over eight years of experimentation at TREC 1
      through 8. Their forms have been abstracted using the
      abbreviations $TF$, $DF$, $LN$ and $QTF$ (term-frequency,
      document-frequency, length-normalization and
      query-term-frequency) to show how these components fit in
      Lucene's term-weight expression.}

    \label{tab:tfxidf}
    
  \end{minipage}

\end{table}

\begin{table}[bht!]
  \centering
  \small
  \begin{minipage}[t]{0.94\textwidth}

    \begin{tabular}{lccccccccccccc}
      \multicolumn{14}{c}{Implementing TF$\times$IDF variants in Lucene}
      \\
      \hline\hline

      Lucene    & $f_{c}(Q,D)$ & $\cdot$  & $f_{q}(Q)$
      & $\cdot$ & $\displaystyle\sum_{T \in Q \cap D}($  & $tf(T)$
      & $\cdot$ & $df(T)$  & $\cdot$  & $f_{b}(T)$
      & $\cdot$ & $f_{n}(T,D)$   & $)$ \\
      
      \emph{BM25}      & $1$          &  $\cdot$ & $1$
      & $\cdot$ & $\displaystyle\sum_{T \in Q \cap D}($  & $TF$
      & $\cdot$ & $DF$          & $\cdot$  & $QTF$
      & $\cdot$ & $1$          & $)$ \\

      \emph{dtb.nnn}   & $1$          & $\cdot$  & $1$
      & $\cdot$ & $\displaystyle\sum_{T \in Q \cap D}($  & $TF$
      & $\cdot$ & $DF$          & $\cdot$  & $QTF$
      & $\cdot$ & $LN$          & $)$ \\

      \hline\hline
    \end{tabular}

    \caption{\small Plugging components of the TF$\times$IDF equation
      into Lucene's scoring equation; the first row is the generalized
      form and the following two rows show the components of two
      popular TF$\times$IDF equations from Table \ref{tab:tfxidf}
      transplanted to Lucene's equation.}

    \label{tab:lucene}

  \end{minipage}
\end{table}

Making a reference to the SIGIR 2012 tutorial on \emph{Experimental
  Methods for Information
  Retrieval}~\cite{Metzler:2012:EMI:2348283.2348534}, Sauparna stated
that we need to take a more rigorous approach to the IR experimental
methodology. A list of best practices were recommended that would add
more structure to IR experiments and prevent the use of systems as
black boxes. These were:

\begin{enumerate}
\item Record test-collection statistics.
\item Provide design documentation for systems.
\item Use a consistent naming scheme and a well-defined notation.
\item Use a evaluation table as a sanity check.
\item Isolate shareable experimental artifacts.
\item Ensure that implementations are traceable to a source in IR
  literature.
\end{enumerate}

In conclusion, Sauparna suggested that if we, the IR research
community, were to build and work with Lucene, it would be helpful to
consider these points when introducing new features into Lucene.
