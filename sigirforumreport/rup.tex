%!TEX root = lucene4IR2016workshop_report.tex
\subsection*{Black Boxes are Harmful}

{\bf Sauparna Palchowdhury (NIST)}: Having seen students and
practitioners in the IR community grapple with abstruse documentation
accompanying search systems and their use as a black box, Sauparna, in
his talk, argued why Lucene is a useful alternative and how and why we
must ensure it does not become another black box. In establishing his
views, he described the pitfalls in an IR experiment and the ways of
mitigation. The suggestions he puts forth, as a set of best practices,
highlights the importance of evaluation in IR to render an experiment
reproducible and repeatable and the need for a well-documented system
with correct implementations of search algorithms traceable to a
source in IR literature. In the absence of such constraints on
experimentation students are misled and learn little from the results
of their experiments and it becomes hard to reproduce the
experiments. As an example, the talk cited a wrong implementation of
the \emph{Okapi BM25} term-weighting equation in a popular research
retrieval system (Table \ref{tab:tfxidf}). Following this was a brief
how-to on implementing \emph{BM25} (or any TF$\times$IDF weighting
scheme) in Lucene (Table \ref{tab:lucene}). This also explained
Lucene's way of computing the similarity between two text documents
(usually referred to as `Lucene's scoring
formula\footnote{\url{https://lucene.apache.org/core/6_2_1/core/org/apache/lucene/search/similarities/TFIDFSimilarity.html}}')
that may be of use to the student and practitioner interested in using
Lucene for IR experiments.

Some of the points of failures mentioned in the talk are misplaced
test-collection pieces (document-query-qrel triplet), counterintuitive
configuration interfaces of systems, poor documentation that makes
systems look enigmatic and lead to the creation of heuristics passed
around by word-of-mouth, naming confusion (a myriad of TF$\times$IDF
model names), blatant bugs and not knowing how the parser works. As
mitigation, Sauparna listed some of the things he did as an
experimenter. He wrote a script (TRECBOX) to abstract the IR
experiment pipeline and map them to configuration end-points of the
three systems; Terrier, Lucene and Indri. This would enable
documenting an experiment's design in plain text files, that could be
shared and the experiment repeated. He constructed a survey of
TF$\times$IDF variants titled \emph{TF$\times$IDF
  Repository}~\cite{rup:TFXIDFRepository} meant to be a single point
of reference to help disambiguate the variants in the wild. All
mentions of term-weighting equations in this repository are traceable
to a source in IR literature. He also shows how to visually juxtapose
evaluation results obtained using a permutation of a set of systems,
retrieval models and test-collections on a chart that would act as a
sanity check for the system's integrity. As a part of these
investigations he modified Lucene for use with TREC collections (the
mod was named LTR) which is available for others to use. The ``mod''
is also accompanied by notes to augment Lucene's documentation. The
gamut of Sauparna's work is to be found online~\cite{rup:IR}.

Lucene's documentation does not use well-defined notation to represent
its way of computing the similarity score between a pair of
documents. The notation below is Lucene's scoring equation, lifted
from Lucene's documentation. It uses actual function names that are to
be found in Lucene's source code;

$score(Q,D) = coord(Q,D) \cdot qnorm(Q) \cdot \displaystyle\sum_{T \in Q} (tf(T \in D) \cdot idf(T)^2 \cdot boost() \cdot norm(T, D))$\\

Sauparna's explanation begins with a well-defined, generalized,
notation for Lucene's scoring in step with the definition from
Lucene's documentation~\cite{Lucene:6.2.1:Scoring};

$score(Q,D) = f_{c}(Q,D) \cdot f_{q}(Q) \cdot \displaystyle\sum_{T \in Q \cap D}(tf(T_{k}) \cdot df(T_{k}) \cdot f_{b}(T_{k}) \cdot f_{n}(T_{k},D)))$\\

He picks two popular TF$\times$IDF variants, breaks them down into
meaningful components (a term-frequency transformation, a
transformation on the document-frequency and a length normalization
coefficient) and plugs these components into Lucene's equation. The
components in Lucene's equation that are left unused are replaced by
the integer $1$, meaning, the functions return $1$; which has no
effect on the chain of multiplications. Table \ref{tab:tfxidf} lists
the variants and components and Table \ref{tab:lucene} shows where the
components were transplanted to.

\begin{table}
  \centering
  \small
  \begin{minipage}[t]{0.65\textwidth}
    
    \begin{tabular}{lcc}
      \multicolumn{3}{c}{TF$\times$IDF Variants: what's correct and what's not.}\\
      \hline\hline
      \\
      Name & $w_{ik}$ & $w_{jk}$\\
      \hline
      \\
      BM25(A)
      & $\frac{f_{ik}}{k_{1}((1-b)+b\frac{dl_{i}}{avdl})+f_{ik}} \times \log(\frac{N-n_{k}+0.5}{n_{k}+0.5})$
      & $\frac{(k_{3}+1)f_{jk}}{k_{3}+f_{jk}}$ \\
      \\
      BM25(B)
      & $\frac{(k_{1}+1)f_{ik}}{k_{1}((1-b)+b\frac{dl_{i}}{avdl})+2f_{ik}} \times \log(\frac{N-n_{k}+0.5}{n_{k}+0.5})$
      & $\frac{(k_{3}+1)f_{jk}}{k_{3}+f_{jk}}$ \\
      \\\hline
      \\
      Okapi BM25
      & $\frac{(k_{1}+1)f_{ik}}{k_{1}((1-b)+b\frac{dl_{i}}{avdl})+f_{ik}} \times \log(\frac{N-n_{k}+0.5}{n_{k}+0.5})$
      & $\frac{(k_{3}+1)f_{jk}}{k_{3}+f_{jk}}$ \\
      \\
      components & $T \times I$ & $Q$ \\
      \\\hline
      \\
      SMART dtb.nnn
      & $\frac{(1+\log(1+\log(f_{ik}))) \times \log(\frac{N+1}{n_{k}})}{1-s+s \cdot \frac{b_{i}}{avgb}}$
      & $f_{jk}$ \\
      \\
      components & $T \times I \div L$ & $Q$ \\
      \\\hline\hline
      \label{tab:tfxidf}
    \end{tabular}
    
    \caption{ The similarity score;
      $score(D_{i},D_{j})=\sum_{k=1}^{t}(w_{ik} \cdot w_{jk})$
      $\forall i \neq j$, combines the weight of a term $k$ over the
      $t$ terms which occur in document $D_{i}$ and $D_{j}$. Since a
      query can also be thought of as a document in the same vector
      space, the symbol $D_{j}$ has been used to denote a query
      without introducing another symbol, say, $Q$. BM25(A) and
      BM25(B) are the two incorrect implementations found in a popular
      retrieval system. Comparing them to \emph{Okapi BM25} on the
      third row shows that A has the $k_{1}+1$ factor missing in the
      numerator, and B uses twice the term-frequency $2f_{ik}$ in the
      denominator. Neither can they be traced to any source in IR
      literature, nor does the system's documentation say anything
      about them. The \emph{Okapi BM25} and the \emph{SMART dtb.nnn}
      variants are known to be effective formulations developed by
      trial and error over eight years of experimentation at TREC 1
      through 8. Their forms have been abstracted using capital
      letters to show how these components fit in Lucene's term-weight
      expression.}

  \end{minipage}
\end{table}

\begin{table}[bht!]
  \centering
  \small
  \begin{minipage}[t]{0.94\textwidth}

    \begin{tabular}{lccccccccccccc}
      \multicolumn{14}{c}{IMPLEMENTING TF$\times$IDF VARIANTS IN LUCENE}
      \\
      \hline\hline

      Lucene    & $f_{c}(Q,D)$ & $\cdot$  & $f_{q}(Q)$
      & $\cdot$ & $\displaystyle\sum_{T \in Q \cap D}($  & $tf(T_{k})$
      & $\cdot$ & $df(T_{k})$  & $\cdot$  & $f_{b}(T_{k})$
      & $\cdot$ & $f_{n}(T_{k}, D_{j})$   & $)$ \\
      
      BM25      & $1$          &  $\cdot$ & $1$
      & $\cdot$ & $\displaystyle\sum_{T \in Q \cap D}($  & $T$
      & $\cdot$ & $I$          & $\cdot$  & $Q$
      & $\cdot$ & $1$          & $)$ \\

      dtb.nnn   & $1$          & $\cdot$  & $1$
      & $\cdot$ & $\displaystyle\sum_{T \in Q \cap D}($  & $T$
      & $\cdot$ & $I$          & $\cdot$  & $Q$
      & $\cdot$ & $L$          & $)$ \\

      \hline\hline
    \end{tabular}

    \caption{\small Plugging components of the TF$\times$IDF equation into
      Lucene's scoring equation; the first row is the generalized form
      and the following two rows show the components of two popular
      TF$\times$IDF equations transplanted to Lucene's equation. Table
      \ref{tab:tfxidf} specifies what the capital letters represent.}

    \label{tab:lucene}

  \end{minipage}
\end{table}

Making a reference to the SIGIR 2012 tutorial on \emph{Experimental
  Methods for Information
  Retrieval}~\cite{Metzler:2012:EMI:2348283.2348534}, Sauparna states
that we need to take a more rigorous approach to the IR experimental
methodology. A list of best practices was recommended that would add
more structure to IR experiments and prevent the use of systems as
black boxes. These were:

\begin{enumerate}
\item Record test-collection statistics.
\item Provide design documentation for systems.
\item Use a consistent naming scheme and a well-defined notation.
\item Build an evaluation table to be used as a sanity-check.
\item Isolate shareable experimental artifacts.
\item Ensure that implementations are traceable to a source in IR
  literature.
\end{enumerate}

In conclusion, Sauparna suggests that if we, the IR research
community, are to build and work with Lucene, then it would be helpful
to consider these points when introducing new features into Lucene.


