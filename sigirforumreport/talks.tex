%!TEX root = lucene4IR2016workshop_report.tex
\section{Keynotes and Invited Talks}
During the course of the workshops a series of talks on how Lucene is being used in Industry, Teaching and for Evaluation along with more technical talks on the inner workings of how Lucene's scoring algorithm works and how learning to rank is being included into Solr, were presented\footnote{\scriptsize{Slides are available from \url{www.github.com/leifos/lucene4ir}}}. A summary of each talk is below.

\subsection*{Introduction Talk: Why are we here?}
{\bf Leif Azzopardi, University of Strathclyde}:
Leif explained how after attending the lively Reproducibility workshop~\cite{arguello2016repro} at ACM SIGIR 2015, he wondered where the Lucene team was, and why, if Lucene and the community is so big, why they don't come to IR conferences - he posited that perhaps we haven't been very inclusive or welcoming to such a large community of search practitioners. He further asserted that this has reduced our capacity to transfer our knowledge and experience into one of the largest Open Source toolkits available. He argued that if we as academics want to increase our impact then we need to improve how we transfer our knowledge to industry. One way is working with large search engines, but what about other industries and organisations that need search and use toolkits like Lucene? He argued that we need to start speaking the same language i.e. work with Lucene et al and look for opportunities on how we can contribute and develop resources for training and teaching IR and how to undertake evaluations and data science using widely used, supported and commonly accepted Open Source toolkits. He described how this workshop was a good starting point and opportunity to explore how academia and industry can better work together, where we can identify common goals, needs and resources that are needed to foster this relationship. 

%!TEX root = lucene4IR2016workshop_report.tex
\subsection*{Keynote Talk: Apache Lucene in Industry} 
{\bf Charlie Hull, FlaxSearch}: In his talk, Charlie first introduced FlaxSearch, and how it evolved over the years. Charlie explained that they have been building search applications using open search software since 2001. Their focus is on building, tuning and supporting fast, accurate and highly scalable search, analytics and Big Data applications. They are partners with Lucid Imagination, leading Lucene specialists and committers. When Lucene first came out clients were at reluctant to adopt open source, but nowadays it has been much more acceptable. Charlie notes that now you don't have to explain what open source, and why it should be used. He described how lucene-based search engines have risen in use - and that search and data analytics are available to those without six figure budgets. Charlie points out that Lucene is appealing because it is the most widely used open source search engine, which is hugely flexible, features rich, scalable and performant. It is supported by a large and healthy community and backed by the Apache Software Foundation. Many of world's largest companies use Lucene including Sony, Siemens, Tesco, Cisco, Linkedin, Wikipedia, WordPress and Hortonworks. Charlie notes that they typically don't use Lucene directly, instead they use the search servers, built on top of Lucene, i.e. Apache Solr (which is mature, stable, and crucially highly scalable), or ElasticSearch (easy to get started with, great analytics, scalable). He contrasts these products with some of the existing toolkits in IR, and remarks, that ``no one in industry has ever heard of them!''. So even though they have the latest research encoded within them, it is not really viable for businesses to adopt them, especially as support for such toolkits are high limited. The research needs to be with Lucene-based search services for it to be used and adopted. 

Charlie then described a number of projects that they have been working on. (1) and (2)..

Based on Charlie's extensive experience he provided us with a number of home truths:
\begin{itemize}
	\item Open source does not mean cheap 
	\item Most Search engines are the same \todo{... in what sense}
	\item Complex features are seldon used - and often confusing
	\item Search testing is rarely comprehensive
	\item Good search developers are hard to find
\end{itemize}


Charlie reflected on these points considering how we can do better. First, learn what works in industry and how industry are using search - there are lots of research challenges which they rarely get to solve and address but solutions to such problems would have real practical value. Second, improve Lucene et al with ideas from academia - faster - for example, it took years before BM25 replaced TFIDF as the standard ranking algorithm, where as  toolkits like Terrier already have infrastructure for Learning to Rank, while this is only just being developed in Lucene. Third, he pointed out that testing and evaluation of Lucene based search engines is very limited, and that thorough evaluations by search developers is poor. He argued that this could be greatly improved, if academics and researchers, contributed to the development of evaluation infrastructure, and transferred their knowledge to practitioners on how to evaluate. Lastly, he pointed that the lack of skilled and knowledgable search developers was problematic - having experience with Lucene, Solr and ElasticSearch are highly marketable skills, especially, when there is a growing need to process larger and larger volumes of data - big data requires data scientists! So there is the pressing need to create educational resources and training material for both students and developers. 



 






%!TEX root = lucene4IR2016workshop_report.tex

\subsection*{Using Lucene for Teaching and Learning IR: The University of Granada case of study} 

{\bf  Prof. Juan M. Fern\'andez-Luna(University of Granada) }: In his talk, Juanma explained the Bologna Process, the establishment of the European Higher Education Area, and how the University of Granada (UGR) has adopted its study programmes, changing the existing undergraduate and master degrees and introducing a few new ones. Currently, Computer Science studies at UGR are composed of an undergraduate degree of four years and a master degree of one, with three different options in this last case: one professional master in Computer Science and two research masters (Data Science and Computer Engineering, and Software Development). With a lot more attention focused on Information Retrieval, a number of courses have been introduced within their undergraduate and masters courses: 

\begin{itemize}
\item {\it Information Retrieval} at the undergraduate Computer Science degree (6 ECTS, 4th year).  The objective of this subject is that the students learn the foundation of IR (document preprocessing, indexing, retrieval models, evaluation and text classification and clustering).

\item {\it Information Management in the Web} at Master in Computer Science (4 ECTS). This subject is composed of three parts: social network analysis, IR and recommender systems. The basic aim is to show the students different ways of managing and accessing the information in the Web. The part related to IR is focused just in briefly explaining the IR foundations. 

\item {\it Information Retrieval and Recommender Systems} at Master in Data Science and Computer Engineering (3 ECTS). In the context of this master, IR foundations are shown to the students, but with a perspective closer to Data Science (preprocessing of large document collections, clustering and classification, etc.).

\item {\it Undergraduate and Master Thesis}  (12 ECTS). Each degree contains a final thesis where the students have to show the skills they have acquired during their studies by means of the development of a project. These are proposed by the lecturers and some of them are related to Information Retrieval.
\end{itemize}

Juanma explained that before introducing programming details, the lecturers at UGR thought that it would be better if  students could understand the core IR process itself. To facilitate the teaching and learning process, {\it SulaIR}~\cite{Fernandez2012} was designed. This is a desktop tool that covers the different IR stages: web crawling, document pre-processing,  indexing, retrieval and relevance feedback. The tool lets students interact with all these processes and secure the concepts from a practical point of view. 

In any of these IR-related subjects, the same question came about when the lecturers were planning 
lab work and exercises, is it better to: (1) create IR projects from the scratch, programming even the more basic classes and focusing on the implementation details, or (2) use an existing IR library (Lucene, Terrier, Lemur, MG, etc.) and focusing on the process?

At the very beginning, they opted for the first alternative as they thought that this could help to understand the details of the search engines, but this created to two learning challenges. The students had to understand the IR process, in a first abstraction step,  and then, to transfer it to code, in a second step. Often the second step interfered with their understanding of the first step as they faced many programming challenges. So the learning process was not very effective.

Therefore, Juanma and the lecturers made the decision of using an Open Source library, where the problem is reduced to learning about an API and identifying the classes and methods that need to be used. In this case, students need not care about the implementation details, per say, instead they could focus on the process (or at least this was their theory). In addition, they realised that the typical professional developer with real needs regarding IR will use an API/Toolkit and there will not be programming IR-related modules from scratch. From all of available APIs, Lucene was, without any doubt, the first choice. This was because it is the most popular IR toolkit openly available and most widely used in industry.

Juanma points out that Teaching and Learning IR with Lucene in these subjects is not without its own problems and challenges. After using it with their courses, the lecturers came to a number of realisations:

\begin{itemize}

\item From {\it Information Retrieval}:
	\begin{itemize}
		\item Lucene is a ``monster'', with lots of classes and methods. This is because it is a large-scale production system, and so students often are frighten by it, unsure of how to work with it.
	\item Considering Java as the programming language to work with Lucene, our students usually have to learn this language first, as they are used to work with C++ in the degree. This language could be a possibility but the Lucene C++ API is poorly documented in comparison to its Java version, so it was discarded.
	\item There is a large amount of documentation about how basic tasks are carried out with Lucene, but sometimes the students select sources from different versions of Lucene. And this causes many headaches when understanding and debugging the problems faced when programming with Lucene. 
	%In addition, the good bibliographic references are not adapted to the Lucene last versions. 
	\item The students' learning curve is very steep when working with the basic process. However, when they progress to more advanced topics such as fielded and complex queries, they find it extremely difficult to progress and really struggle.
\end{itemize}

\item From {\it Information Management in the Web}:

	\begin{itemize}
		\item Due to the length of the course, the ABCs of IR with Lucene are shown to the students. They consider this a good approach because in case of needing to build a search engine in their professional career, they have the basic knowledge to use a toolkit to configure a search engine.
	\end{itemize}

\item From {\it Information Retrieval and Recommender Systems}:

	\begin{itemize}
		\item In the context of a Data Science Master, students are not so interested in the retrieval process itself, but in the pre-processing and indexing stages as bases of further tasks related to machine learning. Other toolkits, as Tika, are shown.
		\item Students have different programming backgrounds (Computer Science, Statistics, Mathematics, Information Science, Electronics, etc.), so it is a real problem to use Lucene for developing lab projects. Considering that R is the programming language on which most of the subjects from the Data Science master are based, an initiative such as RLucene\footnote{https://github.com/s-u/RLucene} could be very interesting for the students.
	\end{itemize}

\item From {\it Undergraduate and Master Thesis}:

	\begin{itemize}
		\item It is difficult to find Lucene advanced documentation for more specific topics, so they usually spend too much time trying to work out how such functionality works (usually through trial and error). In addition,  it is very difficult for them to develop new retrieval models or ranking functions.
		\item As Lucene has not got native support for documents, they have to make great efforts to build the piece of software required to extract the text of the real documents that they find in their projects.
		\item There is no support for IR evaluation in Lucene, so students have to write their own tools to evaluate the performance.
	\end{itemize}

\end{itemize}

As a conclusion, Juanma and the other lecturers at UGR recognised that Lucene is a great tool for teaching and learning IR, but clearly the experience would be much more productive if:

\begin{itemize}
\item Lucene documentation or tutorials from the point of view of teaching and learning were available as well as material describing advanced tasks.
\item Visual and/or Command-Line Tools for teaching and learning the IR process based on Lucene, a kind of SulaIR-L, would really be very useful.
\item Visual and/or Command-Line Tools for IR evaluation (TREC-based) were available, or at least, classes for these purposes were included in the API.
\end{itemize}

%!TEX root = lucene4IR2016workshop_report.tex
\subsection*{Black Boxes are Harmful}

{\bf Sauparna Palchowdhury (National Institute of Standards and
  Technology, Gaithersburg, Maryland, USA)}: Having seen students and
practitioners in the IR community grapple with abstruse documentation
accompanying search systems and their use as a black box, Sauparna, in
his talk, argued why Lucene is a useful alternative and how and why we
must ensure it does not become another black box. In establishing his
views, he described the pitfalls in an IR experiment and the ways of
mitigation. The suggestions he put forth, as a set of best practices,
highlighted the importance of evaluation in IR to render an experiment
reproducible and repeatable and the need for a well-documented system
with correct implementations of search algorithms that are traceable
to a source in IR literature. In the absence of such constraints on
experimentation students are misled and learn little from the results
of their experiments and it becomes hard to reproduce the
experiments. As an example, the talk cited a wrong implementation of
the \emph{Okapi BM25} term-weighting equation in a popular research
retrieval system (Table \ref{tab:tfxidf}). Following this was a brief
how-to on implementing \emph{BM25} (or any TF$\times$IDF weighting
scheme) in Lucene (Table \ref{tab:lucene}). This also explained
Lucene's way of computing the similarity between two text documents
(usually referred to as \emph{Lucene's scoring
  formula}\footnote{\url{https://goo.gl/ZOMVYe}}).

Some of the points of failure mentioned in the talk were misplaced
test-collection pieces (document-query-qrel triplet), counterintuitive
configuration interfaces of systems, poor documentation that make
systems look enigmatic and lead to the creation of heuristics passed
around by word-of-mouth, naming confusion (a myriad of TF$\times$IDF
model names), blatant bugs and a obscure parser. As mitigation,
Sauparna listed some of the things he did as an experimenter. He wrote
a script (TRECBOX\footnote{\url{https://github.com/sauparna/TRECBOX}})
to abstract parts of the IR experiment pipeline and map them to
configuration end-points of the three systems; Terrier, Lucene and
Indri. This would enable documenting and sharing an experiment's
design in plain text files. He constructed a survey of term-weighting
equations titled \emph{TF$\times$IDF
  Repository}\footnote{\url{http://kak.tx0.org/IR/TFxIDF}} meant to be
a single point of reference to help disambiguate the variants in the
wild. All equations mentioned in this repository are traceable to a
source in IR literature. He also showed how to visually juxtapose
evaluation results obtained using a permutation of a set of systems,
retrieval models and test-collections on a chart that would act as a
sanity check for the system's integrity. As a part of these
investigations he modified Lucene for use with TREC collections (the
mod was named LTR\footnote{\url{https://github.com/sauparna/LTR}})
which is available for others to use. The ``mod'' is also accompanied
by notes to augment Lucene's documentation. The gamut of Sauparna's
work is collected on a website\footnote{\url{http://kak.tx0.org/IR}}.

Lucene's documentation does not use a well-defined notation to
represent its way of computing the similarity score between a query
$Q$ and document $D$. Equation \eqref{eqn:Lucene-scoring} denotes
Lucene's scoring formula as described in Lucene's documentation. In
the equation $T$ denotes a term. The functions in order from left to
right on the right-hand-side of the equation is the \emph{coordination
  factor}, \emph{query normalization factor}, \emph{term-frequency
  transformation}, \emph{document-frequency transformation},
\emph{query boost} and \emph{document-length normalization factor}. A
well-defined, generalized, notation for Lucene's scoring in step with
the definition from Lucene's documentation is Equation
\eqref{eqn:Lucene-scoring-generalized} (function names were shortened
appropriately).

\begin{equation}
  score(Q,D) = coord(Q,D) \cdot qnorm(Q) \cdot \displaystyle\sum_{T \in Q} (tf(T \in D) \cdot idf(T)^2 \cdot boost(T) \cdot norm(T,D))
  \label{eqn:Lucene-scoring}
\end{equation}

\begin{equation}
  score(Q,D) = f_{c}(Q,D) \cdot f_{q}(Q) \cdot \displaystyle\sum_{T \in Q \cap D}(tf(T) \cdot df(T) \cdot f_{b}(T) \cdot f_{n}(T,D)))
  \label{eqn:Lucene-scoring-generalized}
\end{equation}

To explain Lucene's scoring, he picked two popular TF$\times$IDF
variants, broke them down into meaningful components (a term-frequency
transformation, a transformation on the document-frequency and a
length normalization coefficient) and plugged these components into
Lucene's equation. The components in Lucene's equation that were left
unused were replaced by the integer $1$, meaning, the functions
returned $1$; which would have no effect on the chain of
multiplications. Table \ref{tab:tfxidf} lists the variants and
components and Table \ref{tab:lucene} shows where the components were
transplanted to.

\begin{table}
  \centering
  \small
  \begin{minipage}[t]{0.65\textwidth}
    
    \begin{tabular}{lcc}
      \multicolumn{3}{c}{TF$\times$IDF Variants: What's correct and what's not.}\\
      \hline\hline
      \\
      Name & $w_{ik}$ & $w_{jk}$\\
      \hline
      \\
      BM25(A)
      & $\frac{f_{ik}}{k_{1}((1-b)+b\frac{dl_{i}}{avdl})+f_{ik}} \times \log(\frac{N-n_{k}+0.5}{n_{k}+0.5})$
      & $\frac{(k_{3}+1)f_{jk}}{k_{3}+f_{jk}}$ \\
      \\
      BM25(B)
      & $\frac{(k_{1}+1)f_{ik}}{k_{1}((1-b)+b\frac{dl_{i}}{avdl})+2f_{ik}} \times \log(\frac{N-n_{k}+0.5}{n_{k}+0.5})$
      & $\frac{(k_{3}+1)f_{jk}}{k_{3}+f_{jk}}$ \\
      \\\hline
      \\
      Okapi BM25
      & $\frac{(k_{1}+1)f_{ik}}{k_{1}((1-b)+b\frac{dl_{i}}{avdl})+f_{ik}} \times \log(\frac{N-n_{k}+0.5}{n_{k}+0.5})$
      & $\frac{(k_{3}+1)f_{jk}}{k_{3}+f_{jk}}$ \\
      \\
      components & $TF \times DF$ & $QTF$ \\
      \\\hline
      \\
      SMART dtb.nnn
      & $\frac{(1+\log(1+\log(f_{ik}))) \times \log(\frac{N+1}{n_{k}})}{1-s+s \cdot \frac{b_{i}}{avgb}}$
      & $f_{jk}$ \\
      \\
      components & $TF \times DF \div LN$ & $QTF$ \\
      \\\hline\hline

    \end{tabular}
    
    \caption{ \small The similarity score;
      $score(D_{i},D_{j})=\sum_{k=1}^{t}(w_{ik} \cdot w_{jk})$
      $\forall i \neq j$, combines the weight of a term $k$ over the
      $t$ terms which occur in document $D_{i}$ and $D_{j}$. Since a
      query can also be thought of as a document in the same vector
      space, the symbol $D_{j}$ denotes a query. BM25(A) and BM25(B)
      are the two incorrect implementations found in a popular
      retrieval system. Comparing them to \emph{Okapi BM25} on the
      third row shows that A has the $k_{1}+1$ factor missing in the
      numerator, and B uses twice the term-frequency, $2f_{ik}$, in
      the denominator. Neither can they be traced to any source in IR
      literature, nor does the system's documentation say anything
      about them. The \emph{Okapi BM25} and the \emph{SMART dtb.nnn}
      variants are known to be effective formulations developed by
      trial and error over eight years of experimentation at TREC 1
      through 8. Their forms have been abstracted using the
      abbreviations $TF$, $DF$, $LN$ and $QTF$ (term-frequency,
      document-frequency, length-normalization and
      query-term-frequency) to show how these components fit in
      Lucene's term-weight expression.}

    \label{tab:tfxidf}
    
  \end{minipage}

\end{table}

\begin{table}[bht!]
  \centering
  \small
  \begin{minipage}[t]{0.94\textwidth}

    \begin{tabular}{lccccccccccccc}
      \multicolumn{14}{c}{Implementing TF$\times$IDF variants in Lucene}
      \\
      \hline\hline

      Lucene    & $f_{c}(Q,D)$ & $\cdot$  & $f_{q}(Q)$
      & $\cdot$ & $\displaystyle\sum_{T \in Q \cap D}($  & $tf(T_{k})$
      & $\cdot$ & $df(T_{k})$  & $\cdot$  & $f_{b}(T_{k})$
      & $\cdot$ & $f_{n}(T_{k}, D_{j})$   & $)$ \\
      
      BM25      & $1$          &  $\cdot$ & $1$
      & $\cdot$ & $\displaystyle\sum_{T \in Q \cap D}($  & $TF$
      & $\cdot$ & $DF$          & $\cdot$  & $QTF$
      & $\cdot$ & $1$          & $)$ \\

      dtb.nnn   & $1$          & $\cdot$  & $1$
      & $\cdot$ & $\displaystyle\sum_{T \in Q \cap D}($  & $TF$
      & $\cdot$ & $DF$          & $\cdot$  & $QTF$
      & $\cdot$ & $LN$          & $)$ \\

      \hline\hline
    \end{tabular}

    \caption{\small Plugging components of the TF$\times$IDF equation
      into Lucene's scoring equation; the first row is the generalized
      form and the following two rows show the components of two
      popular TF$\times$IDF equations transplanted to Lucene's
      equation.}

    \label{tab:lucene}

  \end{minipage}
\end{table}

Making a reference to the SIGIR 2012 tutorial on \emph{Experimental
  Methods for Information
  Retrieval}~\cite{Metzler:2012:EMI:2348283.2348534}, Sauparna stated
that we need to take a more rigorous approach to the IR experimental
methodology. A list of best practices were recommended that would add
more structure to IR experiments and prevent the use of systems as
black boxes. These were:

\begin{enumerate}
\item Record test-collection statistics.
\item Provide design documentation for systems.
\item Use a consistent naming scheme and a well-defined notation.
\item Use a evaluation table as a sanity check.
\item Isolate shareable experimental artifacts.
\item Ensure that implementations are traceable to a source in IR
  literature.
\end{enumerate}

In conclusion, Sauparna suggested that if we, the IR research
community, were to build and work with Lucene, it would be helpful to
consider these points when introducing new features into Lucene.


%!TEX root = lucene4IR2016workshop_report.tex
\subsection*{ Deep Dive into the Lucene Query/Weight/Scorer Java Classes}
{\bf Jake Mannix, Lucidworks}:
In this more technical talk, Jake explained how Lucene scores a query, and what classes are instantiated to support the scoring. Jake described, first, at a high level how to do scoring modification to Lucene-based systems, including some ``Google''-like questions on how to score efficiently. Then, he went into more details about the BooleanQuery class and is cousins, showing where the Lucene API allows for modifications of scoring with pluggable Similarity metrics and even deep inner-loop, where ML-trained ranking models could be instantiated - \emph{if you're willing to do a little work}.


%!TEX root = lucene4IR2016workshop_report.tex
\subsection*{Learning to Rank with Solr} 
{\bf Diego Ceccarelli, Bloomberg}
On day two of the workshop, Diego started his talk by explaining that tuning Lucene/Solr et al. is often performed by ``experts'' who hand tune and craft the weightings used for the different retrieval features. However, this approach is manual, expensive to maintain, and based on intuitive, rather than data. His working goal behind this project was to automate the process. He described how this motivated the use of Learning To Rank, a technique that enables the automatic tuning of an information retrieval system by applying machine learning when estimating parameters. He points out that sophisticated models can make more nuanced ranking decisions than a traditional ranking function when tuned in such a manner. During his talk, Diego presented the key concepts of Learning to Rank, how to evaluate the quality of the search in a production service, and then how the Solr plugin works. At Bloomberg, they have integrated a learning to rank component directly into Solr (and released the code as Open Source), enabling others to easily build their own Learning To Rank systems and access the rich matching features readily available in Solr. 



