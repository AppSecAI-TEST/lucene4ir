%!TEX root = lucene4IR2016workshop_report.tex
\subsection{Teaching and Learning}
\label{sec:teaching}

To seed the discussion for this working group, various members explained how the Information Retrieval course was taught at their institute. As Juanma had already discussed how they teach at the University of Granada (see above), others described their courses and experiences. 

%!TEX root = lucene4IR2016workshop_report.tex
{\bf Emanuele Di Buccio, University of Padova}:  Emanuele described one of the courses taught as part of the Master Degree in Statistical Science at the University of Padua, called {\em Information Systems (Advanced)}\footnote{\scriptsize{The description refers to the course editions in the Academic Years 2011/12-2014/15. The professor in charge was Massimo Melucci.}}.
The course covered both basic IR topics: indexing and retrieval methods, retrieval models, and evaluation, along with more advanced topics such as Web Search or Machine Learning for IR. A detailed description of the course contents can be found in~\cite{Melucci2013}, which is an IR book developed from the experiences teaching the course.
The course was designed so that, for most of the topics, lessons at a theoretical level on a specific topic were followed by a laboratory assignment on that topic. The topics covered in laboratory assignments were: 
\begin{itemize}
	\item creation of a test collection, 
	\item  indexing, 
	\item retrieval, 
	\item relevance feedback, 
	\item  link analysis, 
	\item Learning to Rank, and
	\item Optimization of Ranking functions with parameters. 
\end{itemize}

Emanuele explained that students were asked to propose their own methodology to carry out the laboratory activities. For instance, when considering the topic of ``relevance feedback'', each student could propose their own methodology to perform feedback, e.g. through a query expansion method or term re-weighting. Each assignment, then, involved the experimental evaluation on a shared test collection. Indeed, the objective of the assignments was three-fold:
\begin{enumerate}
	\item to better understand the topic; 
	\item to become familiar in the design and the implementation of experimental methodologies to evaluate methods and/or components and, 
	\item more  generally, to test research hypotheses.
\end{enumerate}

Students were allowed to use a manual approach (when possible), a software library or build their own software modules to achieve the assignment objective; a list of software libraries were provided before the first laboratory assignment to make the students aware of possible options.
However, the adoption of a manual approach for some of the laboratory activities was mandatory. For instance, in the case of the assignment on indexing, the use of a manual approach aimed at a better understanding of the conceptual mechanisms to identify the most effective descriptors to retrieve relevant documents. When the students proposed their own methodology for indexing, they were asked to present their approach as a set of steps that can be automated.

The availability of software libraries or resources to easily use the basic operations is crucial to allow the students to test their methodology with little/less effort. In an edition of the course, a lesson was dedicated to a general introduction to Apache Lucene, where sample code was  provided. Along with Apache Lucene, and introduction to ElasticSearch~\cite{elasticsearch} was also presented, particularly how to index documents, perform retrieval, and how to customize the scoring mechanism via scripting\footnote{\scriptsize{ElasticSearch allows to evaluate a custom score via scripts --- see the {\em Scripting} module. Apache Solr provides similar functionalities via {\em Function Queries}.}} The main reason for the introduction to ElasticSearch was that the students could index, retrieve, and customize the retrieval algorithm -- and therefore test some of their methodologies -- without writing actual code but only through the use of REST requests.

Another aspect Emanuele commented on was the heterogeneous background of the students, and how they came from various disciplines. This was one of the reasons, why they did not restrict the laboratory activities to a single software library and  allowed students to select the tool they felt most comfortable with. While students in Computer Science and Computer Engineering were familiar with Java, students in Statistical Science tended to prefer the R language because it was used in many courses within their course degree. Therefore, one of the resources that could be useful for teaching is a R wrapper for Apache Lucene --- wrappers in other programming languages exist, e.g. PyLucene~\cite{PyLucene} for Python. Software libraries such as ElasticSearch, could be useful tools to support teaching: for instance, they provide functionalities -- in the event of ElasticSearch a REST request -- to display how a specific fragment of text is processed given a pipeline, e.g. a specific tokenizer and a set of filters (lowercase, porter stemming, $\dots$).

%!TEX root = lucene4IR2016workshop_report.tex
{\bf Prof. Krisztian Balog, University of Stavanger}:
The Web Search and Data Mining course is part of the Computer Science master's programme at the University of Stavanger, but it is also offered to (advanced) bachelor students.
The data mining part of the course includes data processing, classification and clustering methods.  The IR part consists of indexing, retrieval models, evaluation, link analysis, query modeling, and entity linking and retrieval.
The course is 6 hours per week, which is divided to 2 lectures (2x2 hours) and a practical session (2 hours).  

Krisztian explained that during the lectures, after presenting the theory, students get a small paper exercise sheet where they need to apply the said theory on toy-sized input data.  Examples of such exercises include constructing an index from some input text, computing term weights and scoring a small number of documents, calculating PageRank scores, etc.  They can use a calculator and/or a spreadsheet program, but the input is simple enough for pen-and-paper. The reference solutions to these exercises are made available after the class.  Student feedback has been very positive; they are really appreciative of this element in the lectures. While completing these exercises, interesting questions (both practical and theoretical) can often spring up and be discussed.
 
The practical sessions involve implementing methods from the lectures and applying them on small (but real) datasets.  Typically it happens on two levels.  First, students need to implement one or two of the simpler methods for a given problem, e.g., decision trees or Naive Bayes for classification. Second, they get to use a ready-made third-party implementation; for the previous example, it would be SVM and Random Forests (from the scikit-learn Python package). For retrieval, ElasticSearch is used; the RESTFul API is well documented and can easily be used from Python (or from any programming language for that matter).  Evaluation is a core element of these exercises, so they need to measure and compare the performance of different approaches according to some metric. 
In order to allow students to focus on the more interesting parts of the problem as opposed to more ``mechanical'' tasks (e.g., reading in data from a file), they get the skeleton of the code along with explanations as an iPython notebook, and they only need to complete the missing parts.

Finally, students have a handful of larger (obligatory) assignments throughout the semester that they need to complete in teams.  These assignments involve larger-scale datasets (note that this scale is still in accordance with academic standards, not with industrial ones).  The assignments are set up as competitions on Kaggle,\footnote{Kaggle in Class (\url{https://inclass.kaggle.com/}) is provided free of charge for academics.} with a (hard) deadline and a minimum performance threshold (e.g., a certain MAP score for a ranking task).  There are no restrictions on the choice of the programming language or libraries used.  The members of the best performing team for each assignment are rewarded with some bonus points that they can ``cash in'' during the final exam.\\


%!TEX root = lucene4IR2016workshop_report.tex
{\bf Martin Halvey, University of Strathclyde}: The Department of Computer and Information Science at the University of Strathclyde has two modules relevant to the discussion. The first module is Information Access \& Mining (ISA) is delivered to final year undergraduates and cover a range of techniques for extracting information from textual and non-textual resources, modelling the information content of resources, detecting patterns within information resources and making use of these patterns. The second is Information Retrieval and Access (IRA) which is delivered to Masters students. This module is a required module for students on Strathclyde's Information \& Library Studies and Information Management Masters Programmes, as well as being an optional module for other Masters students. To offer a contrast to other modules Martin described IRA in detail, as the cohort is different to others described. Typically, with some exceptions, the students do not have experience in programming or mathematics in their undergraduate degree. This presents a number of challenges when teaching some of the core concepts, where the syllabus includes:
\begin{itemize}
    \item Information seeking and behaviour
	\item Indexing
	\item Term weighting
	\item Retrieval models
	\item IR evaluation
	\item Multimedia retrieval
	\item User interfaces and interaction
	\item Web retrieval 
\end{itemize}

Martin explained that in laboratory and tutorial sessions that students were given problems to solve on paper. The intention is that students understand how different concepts, models, evaluation measures etc. work. For some problems students are provided with spreadsheets that automatically calculate some of the equations discussed in lectures so that students can see the relationship between different inputs and outputs. 

Martin outlined how developing some demonstrators using Lucene could replicate what he currently does with spreadsheets, with the benefit being that these demonstrators would be based on a real toolkit and also be more adaptable to use a wider range of retrieval models, evaluation measures etc. There is also the possibility in future years that students will be introduced to tools like Apache Lucene and ElasticSearch in a different module to IRA. Here, it was pointed out by Ian Ruthven, that often these students wont need modify such toolkits, but they will need to know how the systems work, how to configure them, and how to evaluate their configurations choices.



\noindent
{\bf Discussion}:  From the various perspectives, it was clear that there was a number of key concepts that were felt to be fundamental to teaching Information Retrieval. From the discussion it was also clear that the lecturers wanted to give students hands-on experience so that they could see the impact and effect of the different components i.e. what does tokenization and stemming do to the size of vocabulary, the size of the index, and the influence on precision and recall. Also, from the descriptions there was a consensus towards teaching IR in an inquiry led manner - focusing mainly on the science (rather than the engineering). In this way the lectures and course work would be focused on presenting experimental contexts in which the students could go off and conduct experiments to gain insights and understanding into the effect and influence of different factors (rather than just told what would happen). In this way, students would become more scientific in their approach, and know how to conduct an experiment (aim, method, results, conclusion). This was seen to be an important skill to learn, both from a research point of view, but also from a very practical point of view, i.e. many students will become data scientists, and so they need to be trained to be methodical in their approach.

With this in mind, we drew up a table of different facets of the courses, roughly split into Indexing, Retrieval, Analysis and Evaluation (see Table~\ref{tbl_teaching}). We considered the two different levels, high (i.e. using/configuring the apps) and low (i.e. programming and coding algorithms). The reason for considering the different levels, was that different cohorts of students have different skill sets or the focus of the course maybe more oriented towards technical under the hood skills versus high level understanding and usage. At the different levels, we considered the different apps being built during the hackathon along with other apps that would also help support teaching and learning in IR. Table~\ref{tbl_teaching} summarises the different apps and some of the things that we would like students to be able to do with them. For example, consider a lecture on stemming and the following questions: what is the influence of stemming and what are the differences between no stemming, porter stemming and krovetz stemming, in terms of vocabulary and index size, and performance? Having apps that let students easily configure and run different stemmers, then be able to conduct retrieval experiments, measure the performance, and analyze/inspect the index, would then enable them to form their own insights into the effects of stemming. On the other hand, during the retrieval algorithms lectures, different comparisons between models can be made, and if required new algorithms developed. In terms of further analysis, an app (ResultAnalyzerApp) could be used to help analyze the influence of document length normalization - does changing the $b$ parameter in BM25 actually effect the length of documents that are retrieved?
While these are simple examples - it was felt that his on-boarding stage helps students to contextualize and understand the taught material - and enables them to go onto conduct more advanced evaluations. 

\begin{table}
  \centering
  \small
    \begin{tabular}{lll}
\hline
Apps                    & High Level  & Low Level \\
\hline\hline
IndexerApp              & \begin{tabular}[c]{@{}l@{}}Modify how the indexer is performed\\ i.e. different tokenizers, parsers, etc\end{tabular} & Can modify parsers, tokenizers, etc                                                                                                                                 \\ \hline
IndexAnalyzerApp        & Inspect the influence of indexer                                                                                      &                                                                                                                                                                     \\ \hline
RetrievalApp            & \begin{tabular}[c]{@{}l@{}}Try out different retrieval algorithms\\ Change retrieval parameters\end{tabular}          & Implement new retrieval algorithms                                                                                                                                  \\ \hline
ResultAnalyzerApp       & Inspect and analyze the results returned                                                                              & \begin{tabular}[c]{@{}l@{}}Customise the analysis, put out other \\ statistics of interest\end{tabular}                                                             \\ \hline
ExampleApp              &   \multicolumn{2}{l}{                                                                                                                   Examples of how to work with the Lucene index, to make modifications }                                                     \\ \hline
Batch Retrieval Scripts & \begin{tabular}[c]{@{}l@{}}Configure to run a series of standard\\ batch experiments\end{tabular}                     & \begin{tabular}[c]{@{}l@{}} Customize to run specific retrieval\\ experiments\end{tabular}                                                                           \\ \hline
RetrievalShellApp       & n/a                                                                                                                   & \begin{tabular}[c]{@{}l@{}}Implement retrieval algorithms \\ not using Lucene's scorer \\ assuming term independence\end{tabular} \\ \hline
\end{tabular}
\caption{ \small A summary of different apps and what could be varied at different levels.}\label{tbl_teaching}
\end{table}
