%!TEX root = lucene4IR2016workshop_report.tex
\subsection{Teaching and Learning}
\label{sec:teaching}


\todo{Juanma}

\todo{Krisztian}


\todo{Martin}


How do you go about teaching IR? What level?

What kinds of things do you need/want from such resources?


How do we see Lucene fitting in? Benefits to students?


\begin{table}
  \centering
  \small
    \begin{tabular}{lll}
\hline
Apps                    & High Level  & Low Level \\
\hline\hline
IndexerApp              & \begin{tabular}[c]{@{}l@{}}Modify how the indexer is performed\\ i.e. different tokenizers, parsers, etc\end{tabular} & Can modify parsers, tokenizers, etc                                                                                                                                 \\ \hline
IndexAnalyzerApp        & Inspect the influence of indexer                                                                                      &                                                                                                                                                                     \\ \hline
RetrievalApp            & \begin{tabular}[c]{@{}l@{}}Try out different retrieval algorithms\\ Change retrieval parameters\end{tabular}          & Implement new retrieval algorithms                                                                                                                                  \\ \hline
trec\_eval              & Measure the performance                                                                                               &                                                                                                                                                                     \\ \hline
ResultAnalyzerApp       & Inspect and analyze the results returned                                                                              & \begin{tabular}[c]{@{}l@{}}Customise the analysis, put out other \\ statistics of interest\end{tabular}                                                             \\ \hline
ExampleApp              &                                                                                                                       & \begin{tabular}[c]{@{}l@{}}Examples of how to work with the Lucene\\ index, to make modifications\end{tabular}                                                      \\ \hline
Batch Retrieval Scripts & \begin{tabular}[c]{@{}l@{}}Configure to run a series of standard\\ batch experiments\end{tabular}                     & \begin{tabular}[c]{@{}l@{}}Customize to run specific retrieval\\ experiments\end{tabular}                                                                           \\ \hline
RetrievalShellApp       & n/a                                                                                                                   & \begin{tabular}[c]{@{}l@{}}Customize to implement retrieval algorithms \\ outwith the Lucene scorer i.e. a simple scorer \\ assuming term independence\end{tabular} \\ \hline
\end{tabular}
\end{table}
